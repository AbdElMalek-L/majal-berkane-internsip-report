\chapter[Assemblage, tests et optimisation]{Assemblage, tests et optimisation}
\label{cp:assemblage-tests}

{
\parindent0pt

\section{Introduction}
Ce chapitre présente les phases d'assemblage, de test et d'optimisation du système AquaDrone. Ces étapes cruciales permettent de valider le bon fonctionnement du véhicule et d'optimiser ses performances avant la mise en service opérationnelle.

\section{Étude de la motorisation}
La motorisation constitue un élément critique du système, nécessitant une étude approfondie pour assurer des performances optimales dans l'environnement marin.

\begin{block}[note]
Le choix de la motorisation est crucial pour la performance et la fiabilité du véhicule en conditions marines difficiles. Une attention particulière doit être portée à la protection contre la corrosion saline et aux vibrations.
\end{block}

\subsection{Choix des moteurs adaptés à l'environnement marin}
\subsubsection{Critères de sélection}
\begin{itemize}
    \item \textbf{Protection IP} : Niveau de protection contre l'eau et la poussière
    \item \textbf{Matériaux} : Résistance à la corrosion saline
    \item \textbf{Température} : Fonctionnement dans une large gamme de températures
    \item \textbf{Vibrations} : Résistance aux vibrations et aux chocs
\end{itemize}

\subsubsection{Types de moteurs évalués}
\begin{itemize}
    \item \textbf{Moteurs brushless} : Efficacité élevée, maintenance réduite
    \item \textbf{Moteurs à courant continu} : Simplicité, coût réduit
    \item \textbf{Moteurs pas à pas} : Précision de positionnement
\end{itemize}

\subsubsection{Sélection finale}
Le choix s'est porté sur des moteurs brushless étanches IP68, offrant le meilleur compromis entre performance, fiabilité et résistance environnementale.

\subsection{Calcul de poussée et intégration}
\subsubsection{Calculs de poussée}
\begin{itemize}
    \item \textbf{Analyse hydrodynamique} : Calcul des forces de résistance à l'avancement
    \item \textbf{Équilibre des forces} : Poussée nécessaire pour vaincre la résistance
    \item \textbf{Marge de sécurité} : Facteur de sécurité pour les conditions difficiles
\end{itemize}

\subsubsection{Intégration mécanique}
\begin{itemize}
    \item \textbf{Support des moteurs} : Structure robuste et ajustable
    \item \textbf{Alignement des hélices} : Précision de l'alignement pour optimiser l'efficacité
    \item \textbf{Protection} : Carénages et protections contre les impacts
\end{itemize}

\section{Dimensionnement énergétique}
Le dimensionnement énergétique est crucial pour assurer l'autonomie opérationnelle du véhicule et optimiser ses performances.

\subsection{Calcul d'autonomie}
\subsubsection{Consommation des composants}
\begin{itemize}
    \item \textbf{Propulsion} : Consommation des moteurs selon la vitesse et les conditions
    \item \textbf{Électronique} : Consommation des systèmes de contrôle et de communication
    \item \textbf{Capteurs} : Consommation des différents capteurs et instruments
\end{itemize}

\subsubsection{Modèles de consommation}
\begin{itemize}
    \item \textbf{Profils de mission} : Différents scénarios d'utilisation
    \item \textbf{Simulation} : Modélisation de la consommation selon les conditions
    \item \textbf{Validation} : Comparaison avec des mesures réelles
\end{itemize}

\subsection{Batteries et alimentation solaire}
\subsubsection{Sélection des batteries}
\begin{itemize}
    \item \textbf{Technologie} : Batteries lithium-ion pour leur densité énergétique
    \item \textbf{Capacité} : Calcul de la capacité nécessaire pour l'autonomie cible
    \item \textbf{Management} : Système de gestion des batteries (BMS) pour la sécurité
\end{itemize}

\subsubsection{Système solaire}
\begin{itemize}
    \item \textbf{Panels solaires} : Sélection et dimensionnement des panneaux
    \item \textbf{Chargeur} : Système de charge intelligent et protection
    \item \textbf{Intégration} : Positionnement optimal pour maximiser l'exposition
\end{itemize}

\section{Scénarios et protocoles de test}
La validation du système nécessite une approche structurée avec des scénarios de test bien définis et des protocoles rigoureux.

\subsection{Tests en laboratoire}
\subsubsection{Tests des composants}
\begin{itemize}
    \item \textbf{Validation individuelle} : Test de chaque composant séparément
    \item \textbf{Interfaces} : Vérification des communications entre composants
    \item \textbf{Performance} : Mesure des performances selon les spécifications
\end{itemize}

\subsubsection{Tests d'intégration}
\begin{itemize}
    \item \textbf{Modules} : Test de l'intégration des différents modules
    \item \textbf{Système complet} : Test du système dans son ensemble
    \item \textbf{Scénarios} : Simulation de différents modes de fonctionnement
\end{itemize}

\subsection{Tests en conditions réelles}
\subsubsection{Tests en bassin}
\begin{itemize}
    \item \textbf{Manœuvrabilité} : Test des capacités de navigation
    \item \textbf{Stabilité} : Vérification de la stabilité en conditions contrôlées
    \item \textbf{Performance} : Mesure des performances de propulsion
\end{itemize}

\subsubsection{Tests en mer}
\begin{itemize}
    \item \textbf{Environnement réel} : Test dans les conditions marines réelles
    \item \textbf{Autonomie} : Validation de l'autonomie énergétique
    \item \textbf{Communication} : Test des systèmes de communication en conditions réelles
\end{itemize}

\section{Résultats obtenus et analyse des performances}
L'analyse des résultats de test permet d'évaluer les performances du système et d'identifier les axes d'amélioration.

\subsection{Performances de navigation}
\begin{itemize}
    \item \textbf{Précision} : Précision du positionnement et de la navigation
    \item \textbf{Stabilité} : Stabilité du véhicule en conditions variées
    \item \textbf{Manœuvrabilité} : Capacité de manœuvre et de changement de direction
\end{itemize}

\subsection{Performances des capteurs}
\begin{itemize}
    \item \textbf{Précision} : Exactitude des mesures des différents capteurs
    \item \textbf{Fiabilité} : Stabilité des mesures dans le temps
    \item \textbf{Calibration} : Maintien de la calibration en conditions réelles
\end{itemize}

\subsection{Performances de communication}
\begin{itemize}
    \item \textbf{Portée} : Distance maximale de communication fiable
    \item \textbf{Débit} : Capacité de transmission des données
    \item \textbf{Fiabilité} : Stabilité de la connexion en conditions marines
\end{itemize}

\section{Problèmes rencontrés (matériels et logiciels)}
Le développement d'un système complexe comme AquaDrone s'accompagne inévitablement de défis et de problèmes à résoudre.

\subsection{Problèmes matériels}
\subsubsection{Corrosion et étanchéité}
\begin{itemize}
    \item \textbf{Identification} : Détection de points de corrosion sur certains composants
    \item \textbf{Causes} : Exposition prolongée à l'environnement marin
    \item \textbf{Solutions} : Amélioration des protections et sélection de matériaux
\end{itemize}

\subsubsection{Problèmes mécaniques}
\begin{itemize}
    \item \textbf{Vibrations} : Vibrations excessives affectant certains capteurs
    \item \textbf{Usure} : Usure prématurée de certains composants mécaniques
    \item \textbf{Solutions} : Amélioration de la conception et des matériaux
\end{itemize}

\subsection{Problèmes logiciels}
\subsubsection{Communication et synchronisation}
\begin{itemize}
    \item \textbf{Latence} : Délais de communication affectant le contrôle en temps réel
    \item \textbf{Synchronisation} : Problèmes de synchronisation entre différents modules
    \item \textbf{Solutions} : Optimisation des protocoles et amélioration de l'architecture
\end{itemize}

\subsubsection{Gestion des erreurs}
\begin{itemize}
    \item \textbf{Détection} : Identification et gestion des erreurs système
    \item \textbf{Récupération} : Mécanismes de récupération automatique
    \item \textbf{Logging} : Amélioration du système de journalisation
\end{itemize}

\section{Solutions mises en place}
La résolution des problèmes identifiés a nécessité la mise en place de solutions innovantes et l'optimisation de l'existant.

\subsection{Améliorations matérielles}
\begin{itemize}
    \item \textbf{Protection renforcée} : Amélioration de l'étanchéité et de la protection contre la corrosion
    \item \textbf{Matériaux optimisés} : Sélection de matériaux plus résistants
    \item \textbf{Conception améliorée} : Optimisation de la conception pour la robustesse
\end{itemize}

\subsection{Améliorations logicielles}
\begin{itemize}
    \item \textbf{Architecture optimisée} : Refactoring du code pour améliorer les performances
    \item \textbf{Protocoles améliorés} : Optimisation des protocoles de communication
    \item \textbf{Gestion d'erreurs} : Implémentation de mécanismes de récupération robustes
\end{itemize}

\subsection{Procédures et maintenance}
\begin{itemize}
    \item \textbf{Procédures de test} : Standardisation des procédures de test
    \item \textbf{Maintenance préventive} : Mise en place d'un programme de maintenance préventive
    \item \textbf{Formation} : Formation des équipes aux nouvelles procédures
\end{itemize}

\section{Conclusion}
Les phases d'assemblage, de test et d'optimisation ont permis de valider le bon fonctionnement du système AquaDrone et d'identifier les axes d'amélioration. Les solutions mises en place ont considérablement amélioré la robustesse et la fiabilité du système.

Cette expérience constitue une base solide pour les développements futurs et démontre la viabilité de l'approche adoptée pour le développement de véhicules autonomes marins. Les perspectives d'amélioration sont détaillées dans \autoref{cp:conclusion-generale}.

} 