\chapter[Introduction générale]{Introduction générale}
\label{cp:introduction-generale}

{
\parindent0pt

\section{Contexte du projet et importance dans le domaine maritime}
Le domaine maritime représente un enjeu crucial pour l'économie mondiale, la sécurité et la préservation de l'environnement. La surveillance des océans, la gestion des ressources halieutiques et la protection des écosystèmes marins nécessitent des solutions innovantes et adaptées aux défis spécifiques de l'environnement marin.

Les technologies autonomes et la robotique maritime émergent comme des solutions prometteuses pour répondre à ces défis. Les véhicules de surface sans équipage (USV - Unmanned Surface Vehicles) offrent de nouvelles possibilités pour la surveillance maritime, la collecte de données océanographiques et l'assistance à la pêche, tout en réduisant les risques pour les équipages humains.

\begin{block}[note]
Cette introduction présente le cadre général du projet AquaDrone, un véhicule autonome multifonction développé pour répondre aux défis spécifiques de l'environnement maritime marocain et méditerranéen.
\end{block}

\section{Problématique générale}
Malgré les avancées technologiques récentes, les systèmes nautiques actuels présentent plusieurs limitations qui freinent leur adoption généralisée et leur efficacité opérationnelle. Les défis incluent la robustesse face aux conditions marines difficiles, l'autonomie énergétique limitée, la fiabilité des communications en mer et l'intégration harmonieuse de multiples capteurs et systèmes.

La complexité de l'environnement maritime - caractérisé par des conditions météorologiques variables, la corrosion saline, et les contraintes de navigation - nécessite des solutions techniques robustes et adaptées.

\section{Objectifs globaux}
Ce projet vise à développer un véhicule autonome multifonction capable de :
\begin{itemize}
    \setlength{\itemsep}{.375em}
    \item Assurer la surveillance maritime continue et autonome
    \item Collecter des données océanographiques et environnementales
    \item Assister les activités de pêche de manière éthique et durable
    \item Contribuer à la recherche scientifique en milieu marin
    \item Démontrer la viabilité des solutions autonomes en environnement maritime
\end{itemize}

La réalisation de ces objectifs contribuera à l'émergence de nouvelles approches pour la gestion et la protection des espaces maritimes, tout en ouvrant la voie à des applications commerciales et scientifiques innovantes. Pour plus de détails sur l'analyse des besoins, voir \autoref{cp:etude-preliminaire}, et pour l'architecture technique, consulter \autoref{cp:architecture-generale}.

\begin{block}[tip]
Ces objectifs ambitieux nécessitent une approche multidisciplinaire combinant expertise en robotique, en océanographie, en électronique et en mécanique marine.
\end{block}

} 