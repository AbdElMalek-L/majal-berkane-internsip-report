\chapter[Architecture générale et composants]{Architecture générale et composants}
\label{cp:architecture-generale}

{
\parindent0pt

\section{Introduction}
Ce chapitre présente l'architecture générale du système AquaDrone, détaillant l'organisation fonctionnelle et technique, ainsi que les composants matériels et logiciels sélectionnés pour répondre aux exigences définies dans l'analyse des besoins.

\section{Architecture fonctionnelle du système}
L'architecture fonctionnelle du système AquaDrone s'articule autour de plusieurs modules interconnectés, chacun assumant des responsabilités spécifiques dans le fonctionnement global du véhicule.

\begin{block}[note]
Cette architecture modulaire permet une maintenance facilitée et une évolution du système selon les besoins futurs, tout en garantissant la robustesse et la fiabilité nécessaires en environnement maritime.
\end{block}

\subsection{Schéma blocs}
Le système peut être représenté par un schéma fonctionnel organisé en plusieurs niveaux :
\begin{itemize}
    \item \textbf{Niveau perception} : Capteurs et systèmes d'acquisition de données
    \item \textbf{Niveau décision} : Système de contrôle et algorithmes d'intelligence artificielle
    \item \textbf{Niveau action} : Actionneurs et systèmes de propulsion
    \item \textbf{Niveau communication} : Interfaces de communication et transmission de données
\end{itemize}

\subsection{Organigrammes fonctionnels}
L'organisation fonctionnelle suit une approche modulaire, permettant une maintenance facilitée et une évolution du système selon les besoins futurs.

\section{Architecture technique}
L'architecture technique du système AquaDrone repose sur une approche distribuée, avec une répartition claire des responsabilités entre les différents modules.

\subsection{Répartition des modules}
\subsubsection{Module de propulsion}
\begin{itemize}
    \item \textbf{Moteurs électriques} : Propulsion principale et direction
    \item \textbf{Contrôleurs de moteurs} : Gestion de la vitesse et du couple
    \item \textbf{Système de direction} : Contrôle de la trajectoire
\end{itemize}

\subsubsection{Module de commande}
\begin{itemize}
    \item \textbf{Microcontrôleur principal} : Cerveau du système
    \item \textbf{Système de navigation} : Calcul de trajectoire et positionnement
    \item \textbf{Interface utilisateur} : Contrôle et monitoring
\end{itemize}

\subsubsection{Module de communication}
\begin{itemize}
    \item \textbf{Module MLO5} : Communication longue distance
    \item \textbf{Interfaces réseau} : Ethernet, Wi-Fi, radio
    \item \textbf{Système de sécurité} : Chiffrement et authentification
\end{itemize}

\subsubsection{Module de capteurs}
\begin{itemize}
    \item \textbf{Capteurs environnementaux} : Température, salinité, pH
    \item \textbf{Capteurs de navigation} : GPS, boussole, accéléromètre
    \item \textbf{Capteurs de vision} : Caméras et systèmes d'imagerie
\end{itemize}

\section{Composants matériels}
La sélection des composants matériels a été guidée par les exigences de robustesse, de fiabilité et de performance dans l'environnement maritime.

\subsection{Tableau complet des capteurs}
\begin{table}[!htpb]
    \caption{Spécifications des capteurs intégrés dans le système AquaDrone.}
    \label{tab:capteurs}
    \centering
    \begin{tabular}{llll}
        \toprule
        \textbf{Type de capteur} & \textbf{Modèle} & \textbf{Spécifications} & \textbf{Interface} \\ 
        \midrule
        Température & DS18B20 & -55°C à +125°C, ±0.5°C & 1-Wire \\
        Salinité & Atlas Scientific & 0-180 ppt, ±2 ppt & I2C \\
        pH & Atlas Scientific & 0-14 pH, ±0.1 pH & I2C \\
        Oxygène dissous & Atlas Scientific & 0-20 mg/L, ±0.1 mg/L & I2C \\
        GPS & NEO-M8N & Précision 2.5m & UART \\
        Boussole & HMC5883L & ±8 gauss, ±1-2° & I2C \\
        Accéléromètre & MPU6050 & ±2g, ±4g, ±8g, ±16g & I2C \\
        \bottomrule
    \end{tabular}
\end{table}

\subsection{Choix du microcontrôleur (Raspberry Pi 4)}
Le Raspberry Pi 4 a été sélectionné comme microcontrôleur principal pour plusieurs raisons :
\begin{itemize}
    \setlength{\itemsep}{.375em}
    \item \textbf{Performance} : Processeur ARM Cortex-A72 quad-core à 1.5GHz
    \item \textbf{Connectivité} : Ethernet Gigabit, Wi-Fi 802.11ac, Bluetooth 5.0
    \item \textbf{GPIO} : 40 broches GPIO pour l'interface avec les capteurs
    \item \textbf{Support} : Large communauté et documentation abondante
    \item \textbf{Coût} : Rapport performance/prix excellent
\end{itemize}

\begin{block}[tip]
Le choix du Raspberry Pi 4 offre un excellent équilibre entre performance, connectivité et coût, tout en bénéficiant d'un écosystème logiciel mature et d'une large communauté de développeurs.
\end{block}

\subsection{Intégration du module MLO5 et protocole réseau}
Le module MLO5 constitue la solution de communication longue distance du système :
\begin{itemize}
    \item \textbf{Technologie} : Communication par satellite ou radio longue distance
    \item \textbf{Portée} : Couverture mondiale ou régionale selon la configuration
    \item \textbf{Bande passante} : Adaptée aux besoins de transmission des données
    \item \textbf{Fiabilité} : Conçu pour les environnements marins difficiles
\end{itemize}

\section{Outils logiciels et bibliothèques}
Le développement du système AquaDrone s'appuie sur un ensemble d'outils logiciels et de bibliothèques spécialisées.

\subsection{CAO, simulation, traitement de données, communication}
\subsubsection{Conception assistée par ordinateur (CAO)}
\begin{itemize}
    \item \textbf{SolidWorks} : Modélisation 3D et conception mécanique
    \item \textbf{Simulation hydrodynamique} : Analyse des performances en mer
    \item \textbf{Calculs de stabilité} : Vérification de la flottabilité et de la stabilité
\end{itemize}

\subsubsection{Simulation et modélisation}
\begin{itemize}
    \item \textbf{Simulink} : Modélisation des systèmes de contrôle
    \item \textbf{Gazebo} : Simulation de la navigation et des capteurs
    \item \textbf{ROS} : Framework pour la robotique
\end{itemize}

\subsubsection{Traitement de données}
\begin{itemize}
    \item \textbf{Python} : Langage principal pour le traitement des données
    \item \textbf{NumPy/SciPy} : Calculs scientifiques et traitement numérique
    \item \textbf{OpenCV} : Traitement d'images et vision par ordinateur
\end{itemize}

\subsubsection{Communication et réseau}
\begin{itemize}
    \item \textbf{MQTT} : Protocole de communication légère
    \item \textbf{TCP/IP} : Communication fiable pour les données critiques
    \item \textbf{UDP} : Transmission en temps réel pour la vidéo
    \item \textbf{RTSP} : Streaming vidéo pour la surveillance
\end{itemize}

\section{Schéma global du système}
Le schéma global du système AquaDrone illustre l'interconnexion entre tous les modules et composants, montrant le flux de données et de commandes entre les différents éléments du système.

\subsection{Interfaces et connecteurs}
\begin{itemize}
    \item \textbf{Interfaces I2C} : Communication avec la plupart des capteurs
    \item \textbf{Interfaces SPI} : Communication haute vitesse avec certains capteurs
    \item \textbf{Interfaces UART} : Communication série avec le GPS et autres modules
    \item \textbf{Interface Ethernet} : Communication réseau principale
    \item \textbf{Entrées analogiques} : Mesures de tension et de courant
\end{itemize}

\subsection{Flux de données}
Le système suit un flux de données unidirectionnel :
\begin{enumerate}
    \item Acquisition des données par les capteurs
    \item Traitement et validation des données
    \item Prise de décision par le système de contrôle
    \item Exécution des actions par les actionneurs
    \item Transmission des données vers la station distante
\end{enumerate}

\section{Conclusion}
L'architecture générale du système AquaDrone a été conçue pour répondre aux exigences de robustesse, de modularité et de performance dans l'environnement maritime. La sélection des composants matériels et logiciels a été guidée par ces exigences, permettant de créer un système fiable et évolutif.

Cette architecture constitue la base de l'implémentation détaillée du système, qui sera présentée dans \autoref{cp:fonctionnement-detaille}, avec un focus particulier sur le fonctionnement détaillé et l'intégration des différents composants.

} 