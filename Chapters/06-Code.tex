\chapter{Code Listings}
At times, you may want to include source code from your programs and applications within your document. To achieve this, you can use two nested environments: \verb|\begin{listing}| to create a listing with both caption and label, and \verb|\begin{minted}| for code highlighting. \autoref{listing:c-code} provides an example of a source code in C.

\begin{listing}[!htpb]
\begin{minted}{c}
#include <stdio.h>
int main() {
   printf("Hello, World!"); /*printf() outputs the quoted string*/
   return 0;
}
\end{minted}
\caption{Hello World in C}
\label{listing:c-code}
\end{listing}

The code mentioned above was inserted into the document. However, an alternative approach is to input your code from an external file. To do so, you just need to use the command \verb|\inputminted{CODE_LANGUAGE}{FILE}|. Of course, you should place that command inside of the \verb|\begin{listing}| environment. \autoref{listing:octave-code} illustrates an example of Octave source code that has been input from an external file.

\begin{listing}[!htpb]
\inputminted{octave}{Code/BitXorMatrix.m}
\caption{XOR Operation in Octave}
\label{listing:octave-code}
\end{listing}

In some cases, when you simply want to highlight a specific command, it's recommended not to use \verb|listing| or \verb|minted|. Instead, you should utilise the \verb|\verb| command for inline highlighting or the \verb|\begin{verbatim}| environment for longer sections of highlighted code. An example of a lengthy \verb|verbatim| section is provided below, demonstrating how to create a \verb|listing| with an input code:

\begin{verbatim}
\begin{listing}[!htpb]
    \inputminted{CODE_LANGUAGE}{FILE}
    \caption{TEXT}
    \label{TEXT}
\end{listing}
\end{verbatim}

Sometimes it is necessary to display longer code that occupies more than one page. For this purpose, please use the environment \verb|\begin{longlisting}|. This environment will easily break your code into multiple pages for better readability without you worrying about the size of your code. An example is shown below in \autoref{listing:cobol-code}.

\begin{longlisting}
\begin{minted}{cobol}
IDENTIFICATION DIVISION.
PROGRAM-ID. BankingSystem.

DATA DIVISION.
WORKING-STORAGE SECTION.
01 CUSTOMER-RECORD.
   05 CUSTOMER-NAME       PIC X(30).
   05 CUSTOMER-AGE        PIC 99.
   05 CUSTOMER-BALANCE    PIC 9(7)V99.
   05 CUSTOMER-STATUS     PIC X(10).

01 CUSTOMER-COUNT         PIC 9999 VALUE 0.

01 TEMP-VARIABLES.
   05 TEMP-NAME            PIC X(30).
   05 TEMP-AGE             PIC 99.
   05 TEMP-BALANCE         PIC 9(7)V99.
   05 TEMP-STATUS          PIC X(10).

PROCEDURE DIVISION.

    -- Accept customer details from the console
    ACCEPT CUSTOMER-RECORD FROM CONSOLE.
    ADD 1 TO CUSTOMER-COUNT.

    -- Process customer records until 'EXIT' is entered
    PERFORM PROCESS-CUSTOMER-RECORD UNTIL CUSTOMER-NAME = 'EXIT'.

    -- Display total number of customers processed
    DISPLAY 'Total number of customers: ' CUSTOMER-COUNT.

    -- End the program
    STOP RUN.

PROCESS-CUSTOMER-RECORD.
    -- Copy customer details to temporary variables
    MOVE CUSTOMER-NAME TO TEMP-NAME.
    MOVE CUSTOMER-AGE TO TEMP-AGE.
    MOVE CUSTOMER-BALANCE TO TEMP-BALANCE.
    MOVE CUSTOMER-STATUS TO TEMP-STATUS.

    -- Display customer details
    DISPLAY 'Name: ' TEMP-NAME.
    DISPLAY 'Age: ' TEMP-AGE.
    DISPLAY 'Balance: ' TEMP-BALANCE.
    DISPLAY 'Status: ' TEMP-STATUS.

    -- Accept next customer record
    ACCEPT CUSTOMER-RECORD FROM CONSOLE.
    ADD 1 TO CUSTOMER-COUNT.
END PROGRAM BankingSystem.
\end{minted}
\caption{COBOL Code for a Basic Banking System}
\label{listing:cobol-code}
\end{longlisting}

