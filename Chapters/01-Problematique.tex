\chapter[Problématique]{Problématique}
\label{cp:problematique}

{
\parindent0pt

\section{Limites des systèmes nautiques actuels}
Les systèmes nautiques traditionnels et même les solutions semi-autonomes actuelles présentent plusieurs limitations significatives qui limitent leur efficacité et leur adoption dans le domaine maritime :

\begin{block}[note]
L'identification de ces limitations constitue la base de la \gls{conception} du système AquaDrone, permettant de développer des solutions innovantes répondant aux défis spécifiques de l'environnement maritime.
\end{block}

\subsection{Limitations technologiques}
\begin{itemize}
    \setlength{\itemsep}{.375em}
    \item \textbf{Autonomie énergétique limitée} : La plupart des systèmes actuels nécessitent des recharges fréquentes ou des interventions humaines
    \item \textbf{Fragilité face aux conditions marines} : Résistance insuffisante à la corrosion saline, aux vagues et aux conditions météorologiques extrêmes
    \item \textbf{Capacités de communication restreintes} : Portée limitée et fiabilité réduite en environnement maritime
\end{itemize}

\subsection{Limitations opérationnelles}
\begin{itemize}
    \setlength{\itemsep}{.375em}
    \item \textbf{Coûts d'exploitation élevés} : \gls{maintenance} fréquente et nécessité d'équipages qualifiés
    \item \textbf{Flexibilité limitée} : Difficulté d'adaptation aux missions variées et aux conditions changeantes
    \item \textbf{Intégration complexe} : Problèmes de compatibilité entre différents systèmes et capteurs
\end{itemize}

\section{Enjeux de la surveillance maritime et de la télédétection}
La surveillance maritime et la télédétection océanographique présentent des défis spécifiques qui nécessitent des solutions innovantes :

\subsection{Enjeux environnementaux}
\begin{itemize}
    \item \textbf{Surveillance continue} : Nécessité de collecter des données 24h/24 et 7j/7 pour comprendre les dynamiques océaniques
    \item \textbf{Couverture géographique étendue} : Besoin de surveiller de vastes zones maritimes avec une résolution temporelle et spatiale élevée
    \item \textbf{Préservation des écosystèmes} : Équilibre entre collecte de données et minimisation de l'impact environnemental
\end{itemize}

\subsection{Enjeux technologiques}
\begin{itemize}
    \item \textbf{Fiabilité des capteurs} : Maintien de la précision des mesures dans des conditions marines difficiles
    \item \textbf{Traitement des données} : Gestion de grands volumes de données en temps réel
    \item \textbf{Calibration et \gls{validation}} : Maintien de la qualité des données sur le long terme
\end{itemize}

\section{Besoins spécifiques d'un véhicule autonome multifonction}
Le développement d'un véhicule autonome multifonction pour l'environnement maritime répond à des besoins spécifiques et critiques :

\subsection{Besoins fonctionnels}
\begin{itemize}
    \item \textbf{Polyvalence} : Capacité d'adaptation à différentes missions (surveillance, pêche assistée, télédétection)
    \item \textbf{Autonomie décisionnelle} : Capacité de prise de décision en temps réel face aux conditions changeantes
    \item \textbf{Coopération multi-véhicules} : Possibilité de travailler en flotte pour des missions complexes
\end{itemize}

\subsection{Besoins techniques}
\begin{itemize}
    \item \textbf{Robustesse} : Résistance aux conditions marines extrêmes et durabilité à long terme
    \item \textbf{Efficacité énergétique} : \gls{optimisation} de la consommation pour maximiser l'autonomie
    \item \textbf{Modularité} : \gls{architecture} permettant l'adaptation et l'évolution selon les besoins
    \item \textbf{Sécurité} : Systèmes de sécurité et de récupération en cas de défaillance
\end{itemize}

\subsection{Besoins opérationnels}
\begin{itemize}
    \item \textbf{Facilité d'utilisation} : Interface simple pour les opérateurs non-experts
    \item \textbf{\gls{maintenance} préventive} : Capacité de diagnostic et d'alerte précoce
    \item \textbf{Conformité réglementaire} : Respect des normes maritimes et environnementales
\end{itemize}

La réponse à ces besoins nécessite une approche intégrée combinant expertise en robotique, en océanographie, en électronique et en mécanique marine, dans le cadre d'un projet innovant et ambitieux. L'\gls{architecture} technique détaillée est présentée dans \autoref{cp:architecture-generale}.

} 