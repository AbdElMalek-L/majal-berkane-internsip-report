\thispagestyle{plain}


% English Abstract 
\pdfbookmark[1]{Abstract}{abstract}
\chapter*{Abstract}
This internship report presents the design and development of AquaDrone, an autonomous, connected, and multifunctional maritime surface vehicle (USV) developed during a PFA internship at Majal Berkane. The project addresses critical challenges in maritime surveillance, oceanographic data collection, and sustainable fishing assistance through innovative autonomous technology.

The research focuses on overcoming limitations of current maritime systems, including limited energy autonomy, vulnerability to marine conditions, and restricted communication capabilities. The proposed solution integrates a modular architecture combining Raspberry Pi 4 control systems, advanced environmental sensors (temperature, salinity, pH, dissolved oxygen), GPS navigation, and robust MLO5 communication modules.

Key achievements include the successful integration of multi-sensor systems, development of autonomous navigation algorithms, and validation of the system in real marine conditions. The project demonstrates the viability of autonomous solutions for maritime environments while contributing to Morocco's technological advancement in marine robotics.

The work establishes a foundation for future developments in autonomous maritime vehicles, offering potential applications in environmental monitoring, scientific research, and commercial maritime operations. This project represents a significant step toward sustainable and efficient maritime operations in the Mediterranean region.

\keywordsen{Autonomous Maritime Vehicle, USV, Marine Robotics, Environmental Monitoring, IoT, Raspberry Pi, Oceanographic Sensors, Autonomous Navigation, Marine Communication Systems.}
\MediaOptionLogicBlank

% French Abstract 
\pdfbookmark[1]{Résumé}{resume}
\chapter*{Résumé}
Ce rapport de stage présente la conception et le développement d'AquaDrone, un véhicule maritime de surface autonome, connecté et multifonctionnel (USV) développé lors d'un stage PFA au sein de Majal Berkane. Le projet aborde les défis critiques de la surveillance maritime, de la collecte de données océanographiques et de l'assistance à la pêche durable grâce à une technologie autonome innovante.

La recherche se concentre sur le dépassement des limitations des systèmes maritimes actuels, incluant l'autonomie énergétique limitée, la vulnérabilité aux conditions marines et les capacités de communication restreintes. La solution proposée intègre une architecture modulaire combinant des systèmes de contrôle Raspberry Pi 4, des capteurs environnementaux avancés (température, salinité, pH, oxygène dissous), la navigation GPS et des modules de communication MLO5 robustes.

Les réalisations principales incluent l'intégration réussie de systèmes multi-capteurs, le développement d'algorithmes de navigation autonome et la validation du système en conditions marines réelles. Le projet démontre la viabilité des solutions autonomes pour les environnements maritimes tout en contribuant à l'avancement technologique du Maroc dans le domaine de la robotique marine.

Le travail établit une base pour les développements futurs de véhicules maritimes autonomes, offrant des applications potentielles dans la surveillance environnementale, la recherche scientifique et les opérations maritimes commerciales. Ce projet représente une étape significative vers des opérations maritimes durables et efficaces dans la région méditerranéenne.

\keywordsen{Véhicule Maritime Autonome, USV, Robotique Marine, Surveillance Environnementale, IoT, Raspberry Pi, Capteurs Océanographiques, Navigation Autonome, Systèmes de Communication Marine.}
\MediaOptionLogicBlank

% Arabic Abstract 
% \pdfbookmark[1]{ملخص}{abstract-ar}
% \chapter*{ملخص}
% في قسم \textit{الملخص}، قدّم ملخصًا موجزًا لمشروعك مع إبراز النقاط الرئيسية. ابدأ ببيان مختصر للمشكلة أو الهدف، يليه وصف لمنهجيتك أو طريقة بحثك. لخّص النتائج أو الاستنتاجات الرئيسية مع التأكيد على أهميتها أو تداعياتها. اختتم بجملة أو جملتين حول الإسهام العام أو تأثير عملك. يجب أن يكون الملخص واضحًا وموجزًا (150-250 كلمة مثاليًا) لتمكين القراء من فهم عملك وأهميته بسرعة.
% \keywordspt{كلمة مفتاحية أ، كلمة مفتاحية ب، كلمة مفتاحية ج.}
% \MediaOptionLogicBlank