\thispagestyle{plain}


% English Abstract 
\pdfbookmark[1]{Abstract}{abstract}
\chapter*{Abstract}
\guideinfo{In the \textit{Abstract} section, provide a concise summary of your project, highlighting the key points. Begin with a brief statement of the problem or objective, followed by a description of your approach or methodology. Summarise the main results or findings, emphasising their significance or implications. Conclude with a sentence or two on the overall contribution or impact of your work. Keep the abstract clear and focused, ideally within 150-250 words, to give readers a quick understanding of your research and its importance.}
\keywordsen{Keyword A, Keyword B, Keyword C.}
\MediaOptionLogicBlank

% French Abstract 
\pdfbookmark[1]{Résumé}{resume}
\chapter*{Résumé}
\guideinfo{Dans la section \textit{Résumé}, présentez un résumé concis de votre projet en mettant en avant les points clés. Commencez par une brève déclaration du problème ou de l'objectif, suivie d'une description de votre approche ou méthodologie. Résumez les principaux résultats ou conclusions en soulignant leur importance ou leurs implications. Concluez par une ou deux phrases sur la contribution globale ou l'impact de votre travail. Le résumé doit être clair et concis (idéalement 150-250 mots) pour permettre aux lecteurs de comprendre rapidement votre travail et son importance.}
\keywordsen{Mot-clé A, Mot-clé B, Mot-clé C.}
\MediaOptionLogicBlank

% Arabic Abstract 
% \pdfbookmark[1]{ملخص}{abstract-ar}
% \chapter*{ملخص}
% \guideinfo{في قسم \textit{الملخص}، قدّم ملخصًا موجزًا لمشروعك مع إبراز النقاط الرئيسية. ابدأ ببيان مختصر للمشكلة أو الهدف، يليه وصف لمنهجيتك أو طريقة بحثك. لخّص النتائج أو الاستنتاجات الرئيسية مع التأكيد على أهميتها أو تداعياتها. اختتم بجملة أو جملتين حول الإسهام العام أو تأثير عملك. يجب أن يكون الملخص واضحًا وموجزًا (150-250 كلمة مثاليًا) لتمكين القراء من فهم عملك وأهميته بسرعة.}
% \keywordspt{كلمة مفتاحية أ، كلمة مفتاحية ب، كلمة مفتاحية ج.}
% \MediaOptionLogicBlank